% PARTIE 3 - PERSPECTIVES %%%%%%%%
% CHAPITRE 6 %%%%%%%%%%%%%%%%%%%%%
%%%%%%%%%%%%%%%%%%%%%%%%%%%%%%%%%%

Nous terminons ce mémoire par un chapitre d'ouverture qui est l'occasion de discuter des perspectives offertes par le projet Richelieu. Le chapitre réunit des pistes de réflexion pour valoriser, prolonger et questionner le projet, tant en termes de médiation auprès du public que pour sa contribution à la discipline de l'histoire de l'art.

%%%%%%%%%%%%%%%%%%%%%%%%%%%%%%%%%%%%%
% SECTION %%%%%%%%%%%%%%%%%%%%%%%%%%%
\section{Des médiations pour le grand public : une valorisation du projet}
Les perspectives de valorisation peuvent être multiples, et leur réalisation sont  l'occasion de sensibiliser le grand public d'une part, et, d'autre part d'assurer une forme de pérennité au projet.

En effet, exposer l'histoire du quartier via une application est une forme de médiation parmi d'autres pour valoriser les recherches. Par exemple, les équipes ont participé à de nombreuses communications scientifiques pour accompagner le projet. Les présentations et les conférences, tout comme les publications dans des revues spécialisées, diffusent les recherches. Cependant, ces formes de médiation restent principalement destinées à un public initié, généralement composé de spécialistes des disciplines concernées.

Rappelons que les rapports du projet nous informent que celui-ci est à destination d'un grand public. Les outils développés offriront \enquote{des ressources aux professionnels du patrimoine et de la recherche ainsi qu'à différents publics (scolaire, amateurs, élus)}.\footcite{BADULESCUpublics2023} Cette orientation du projet soulève une question d'accessibilité. Comment garantir l'accès du projet au grand public ? Il est reconnu qu'une portion du grand public, souvent amateur, n'a pas accès à Internet ou n'est pas familière avec l'utilisation d'un ordinateur \footcite{GRADOZfracture2019} pour consulter la plateforme du projet. Limiter la médiation du projet à l'application reviendrait à exclure une partie du public ou admettre que les objectifs visés ne sont pas réalistes. Toutefois, nous avons identifié deux possibles formes de médiation qui seraient de réels leviers pour mener à bien le projet ambitieux de valorisation destiné au grand public : une exposition muséale et une borne interactive intégrée à celle-ci.

\subsection{Une exposition temporaire au musée}
Un travail de recherche peut parfois être l'objet d'une exposition au musée. Dans certains cas, il s'agit même d'une consécration démocratique qui rend les résultats de recherche accessibles à un public diversifié. Nous pensons que l'exposition temporaire offrirait plusieurs atouts pour valoriser le projet Richelieu et assurer une réception pérenne auprès du public. 

\subsubsection{Des œuvres : medium du passé au présent}
Le corpus constitué par le projet réunit de nombreuses œuvres mais dont il est aisé d'en extraire un échantillon pour représenter les différentes approches de recherche du projet. La nature du corpus, peintures, estampes, et photographies, constitue un ensemble de media aisé à accrocher. Bien que les cartes et plans se distinguent par leur potentielle fragilité, leur présence au sein d'une exposition est souvent source de fascination. Cette exposition pourrait être de taille moyenne, à l'échelle d'une exposition-dossier, plutôt que de la rétrospective. Les thématiques du projet seraient réparties par murs ou salle de l'exposition. Elles donneraient à voir l'histoire d'un bâtiment et toutes ses représentations, le développement de la mode et de ses productions dans le quartier, ou les différentes activités menées au fil des siècles au sein du Palais Royal. Surtout, le lieu de l'exposition devrait être directement lié au quartier : les salles offertes par le quadrilatère Richelieu sont la meilleure perspective qui puisse s'offrir au public. Le voyage dans le passé via l'exposition serait complété avec la balade \textit{in situ} : de la visite réelle dans les galeries voisines à leur visualisation historique au musée. Offrir un tel panorama au public constituerait un aboutissement significatif pour le projet.

\subsubsection{Des publics}
Les publics des musées sont variés et leur fréquentation également. Il ne s'agit pas ici de proposer une étude des publics. Il convient simplement de justifier la raison de l'exposition pour un public dit \enquote{\textit{traditionnel} qui expriment une certaine défiance vis-à-vis de l’interposition d’un écran sur l’expérience esthétique et n’envisagent la visite en ligne qu’en complémentarité avec une visite sur site }\footcite{BADULESCUpublics2023}. En effet, même si une partie du public, majoritairement âgé, est souvent démuni face à l'ère du tout numérique, une exposition du projet lui permettrait d'accéder tout de même au propos du projet Richelieu. Des publics non valides aux handicaps divers tel que les non-voyants et déficients visuels pourraient également accéder au propos du projet via l'exposition qui offre également d'autres atouts avec notamment l'impression en \acrshort{3d} d'un bâtiment, d'une œuvre, voire du quartier à toucher. Il s'agit ici de pratiques communes à multiplier aujourd'hui au musée pour un élargissement massif du public\footcite{WINDHAGERVisualizing2018}. 

\subsection{Une borne interactive permanente}
Une autre médiation à laquelle nous pensons est la visualisation du projet via une borne interactive qui semble être une stratégie plus efficace et rapide à concevoir que l'exposition. En effet, l'application telle qu'elle a été conçue sera consultable sur un écran d'ordinateur mais aussi sur les medium tactiles, qu'il s'agisse d'un téléphone ou d'une tablette. La borne interactive s'intègre à ces mediums de consultation. De plus, les musées diversifient leurs expériences de visite via l'introduction de différents supports numériques où le visiteur devient \enquote{outillé}\footcite{BADULESCUpublics2023}.
Introduire la borne est aussi un excellent support de production d'un matériel éducatif à visée pédagogique. Cette borne serait à installer directement dans l'exposition ou de façon plus pérenne au sein de salle Ovale de la \acrshort{bnf} car elle est ouverte et gratuite au public. Surtout, elle propose déjà de nombreux dispositifs multimédia et tactiles de médiation patrimoniale. Envisager une telle médiation demanderait aussi de traduire les informations en plusieurs langues pour s'assurer que le public étranger puisse découvrir l'histoire du quartier. 
Enfin, exposer l'application via une borne interactive ne rentre pas en contradiction avec la conception traditionnelle du musée : \enquote{On peut aujourd’hui affirmer qu’il n’y a pas de  concurrence entre le musée physique et le musée en ligne, mais plutôt une complémentarité et de nouveaux usages.}\footcite{DELMAS-GLASSHumanites2021} 

Toutefois, de telles médiations pourraient soulever des questions juridiques. En effet, comment justifier de l'exposition numérique de certaines œuvres provenant d'institutions étrangères via le \textit{\acrshort{iiif}} dans une borne interactive sans que cela ne fasse l'objet d'un prêt d'exposition, parfois accompagné de frais de diffusion?  

Finalement, le projet Richelieu offre plusieurs perspectives de valorisation pour un public élargi. Par la même occasion, elle prennent activement part aux médiations contemporaines au sein des musées, participant également à la redéfinition de ceux-ci.

%%%%%%%%%%%%%%%%%%%%%%%%%%%%%%%%%%%%%
% SECTION %%%%%%%%%%%%%%%%%%%%%%%%%%%
\section{Les nouveaux contours de l'histoire de l'art}
Le projet Richelieu ouvre aussi sur de nouvelles perspectives qui dépassent son cadre initial. Conduit par un consortium culturel regroupant des institutions patrimoniales représentatives de l'histoire de l'art en tant que discipline, il s'appuie également sur plusieurs autres domaines pour sa réalisation, notamment la géographie et l'informatique, qui en constituent des composantes majeures. Une question se pose quant à l'impact de cette interdisciplinarité sur l'histoire de l'art : en mobilisant des approches extérieures, cette démarche conduit-elle à redéfinir les contours de la discipline et à en assouplir les frontières ?

\subsection{Définir l'interdisciplinarité}
Que signifie l'interdisciplinarité dans un projet de recherche ? Des chercheurs qui eux-mêmes combinent plusieurs concepts issus de différents domaines de recherche posent la question et proposent des définitions sur lesquelles nous nous basons pour cette partie\footcite{STEMBERAdvancing1991, JENSENIUSSound2022}. 

\subsubsection{Définition et théorie}
L'interdisciplinarité est souvent confondue avec d'autres approches analogues mais pourtant bien distinctes selon  Stember, chercheuse en sciences sociales ayant théorisé cette notion dans les années 1990. Par exemple, la multidisciplinarité désigne une approche où des chercheurs issus de différentes disciplines travaillent ensemble, chacun s'appuyant sur les connaissances issues de leurs disciplines respectives. Au contraire, la trandisciplinarité crée une unité de cadre intellectuel (\textit{intellectual frameworks}) qui va au-delà des perspectives disciplinaires respectives. L'interdisciplinarité est à la jointure de ces typologies : elle consiste à intégrer les connaissances et les méthodes de différentes disciplines, en utilisant une véritable approche synthétique. Toutes ces étapes se succèdent pour retourner à la conception d'une discipline puis à la transdiciplinarité qui consiste à considérer une discipline du point de vue d'une autre. Telle est l'évolution des disciplines analysée par les chercheurs, à l'image d'une boucle comme l'illustre le schéma~\ref{fig:interdisciplinarite}.

\begin{figure}[ht!]
    \centering
    \includegraphics[width=0.8\linewidth]{images/interdisciplinarités.png}
    \caption{Schéma des différents niveaux de disciplinarité réalisé par Jensenius (2022) à partir des travaux de Stember (1991) et Zigler (1990).}
    \label{fig:interdisciplinarite}
\end{figure}

Selon cette conception de l'interdisciplinarité, c’est en observant et en intégrant les méthodes, concepts et théories d'autres disciplines que celles-ci convergent pour former une approche unifiée. Stember avance plusieurs arguments pour justifier cette transformation. Le premier argument est de nature intellectuelle : lorsqu'une discipline se spécialise en sous-domaines, elle tend à s'isoler des autres. Pour éviter cet isolement, l'apport d'autres disciplines devient nécessaire, modifiant ainsi ses frontières. Le second argument est d'ordre pratique et repose sur l'idée que la structure du monde, tel qu'il est étudié, ne correspond pas aux divisions académiques. Un sujet peut relever de plusieurs domaines à la fois. Pour appuyer son argumentation, la chercheuse mentionne plusieurs exemples, dont l'ingénierie, qui se distingue en tant que discipline autonome tout en intégrant des savoirs issus des mathématiques, de la physique et de la mécanique. De même, la discipline infirmière requiert des connaissances provenant de divers champs tels que la biologie, la chimie, la sociologie et la psychologie. L'histoire des sciences démontre que les disciplines sont nombreuses à franchir leurs frontières traditionnelles alors établies au XIX\ieme~ siècle. En quoi le destin de l'histoire de l'art serait-il différent ? Nous prendrons le projet Richelieu comme cas d'étude pour proposer une réponse.

\subsection{Le cas du projet Richelieu}
A partir de la théorie et de la définition précédente, nous avons schématisé la place occupée par le projet Richelieu dans ce vaste paysage interdisciplinaire (voir la figure \ref{fig:transdis}).

\begin{figure}[ht!]
    \centering
    \includegraphics[width=0.8\linewidth]{images/transdisciplinarité_richelieu.png}
    \caption{Diagramme de Venn de la trandisciplinarité du projet Richelieu, \mhd.}
    \label{fig:transdis}
\end{figure}

Ce diagramme de Venn ne tend pas à l'exhaustivité historique et panoramique et pourrait même être considéré comme réducteur, car il n'inclut pas d'autres disciplines avec lesquelles l'histoire de l'art interagit, telles que la littérature, l'archéologie, l'architecture, l'esthétique, la philosophie, la psychologie, la sociologie, la chimie ou l'économie. Mais il a le mérite de réunir les différents domaines dans lesquels évolue le projet Richelieu et qui ont été cités au cours de ce mémoire, notamment la cartographie Web. Le projet Richelieu est symbolisé par le point noir au centre de ce nœud gordien. 

Issue de l'histoire et des arts\footcite{ALTETLetude2023}, l'histoire de l'art telle qu'elle est conçue dans le projet tend à devenir géographique d'une part et numérique et informatique d'autre part. Ces deux parties sont respectivement symbolisées par ce qu'on appelle les \textit{Spatial humanities} et les \textit{Digital humanities} -- tous deux issus de tournants historiographiques communs. 

\subsubsection{Les tournants historiographiques} 
Ce terme \enquote{tournant} décrit une rupture entre divers courants de pensée et est principalement issu d'une littérature anglophone qui émerge dans les années 1980-1990 et qui continue de questionner aujourd'hui\footcite{CLAVALgeographie2008, CLAVALOu2008, DOMINICLARAMEEtournants2017, Humanites}. Par le \textit{spatial turn}, \enquote{on désigne la propension grandissante des historiens à prêter attention à la dimension spatiale dans l’étude du passé, à partir des années 1990, et à tisser en conséquence des rapports toujours plus étroits avec les spécialistes de géohistoire et de géographie culturelle.}\footcite{TORREtournant2008} L'espace n'est plus envisagé d'un point de vue ontologique, mais comme un processus intégrant divers phénomènes, notamment ceux liés aux interactions d'échelle\footcite{BESSEApproches2010}. En histoire de l'art, les théories de transferts culturels (M. Espagne, 1999\footcite{ESPAGNEnotion2013}) et de géographie de l'art (T. DaCosta Kaufman, 2004\footcite{KAUFMANNGeography2004}) constituent des approches précurseures. Elles se révèlent heuristiques quand entre en jeu le tournant numérique. Le \textit{Digital turn} englobe plus largement le domaine des sciences humaines et sociales, historiquement appelées les \enquote{humanités}\footcite{MOUNIERhumanites2018, SAINT-RAYMONDDans2024}. Il désigne depuis une dizaine d'années l'intégration des outils numériques, telle que la numérisation des sources, et des méthodes computationnelles telle que la \textit{Computer Vision}, dans la recherche et l'analyse des phénomènes culturels. Ce tournant transforme également les approches traditionnelles d'utilisation des données par les utilisateurs car il ouvre de nouvelles perspectives de recherche sur l'exploitation des corpus de sources primaires et secondaires et sur la visualisation de l'information. 

Il convient de se demander si, dans le cadre du projet Richelieu, la convergence de ces deux tournants historiographiques entraîne une transformation épistémologique de l'histoire de l'art portée par les \textit{Digital humanities} (\acrshort{dh}) et les \textit{Spatial humanities} (\acrshort{sh}).

\subsubsection{Un enjeu épistémologique}

Définissons les termes pour comprendre en quoi l'histoire de l'art se transformerait-elle. Les humanités numériques (\acrshort{dh}) se distinguent des outils informatiques tels que Zotero ou les logiciels de traitement de texte, utilisés dans divers domaines bien au-delà du stricte cadre universitaire, qui servent principalement de supports auxiliaires pour accomplir des tâches. Un ensemble de chercheurs des \acrshort{dh} proposent une définition au sein d'un manifeste\footnote{Le Manifeste des digital humanities a été rédigé en 2010 lors
du ThatCamp de Paris, publié en 2010 dans le Journal des anthropologues, mis à jour en 2012, et cosigné par plus de 250 chercheurs et 10 institutions.} de référence dans lequel il est communément admis que : 
\begin{displayquote}
    \begin{enumerate}
        \item Le tournant numérique pris par la société modifie et interroge les conditions de production et de diffusion des savoirs.
        \item Les \textit{Digital humanities} concernent l’ensemble des Sciences humaines et sociales, des Arts et des Lettres. Les \textit{Digital humanities} ne font pas table rase du passé. Elles s’appuient, au contraire, sur l’ensemble des paradigmes, savoir-faire et connaissances propres à ces disciplines, tout en mobilisant les outils et les perspectives singulières du champ du numérique.
        \item Les \textit{Digital humanities} désignent une transdiscipline, porteuse des méthodes, des dispositifs et des perspectives heuristiques liés au numérique dans le domaine des Sciences humaines et sociales.
    \end{enumerate}
\end{displayquote}
En ce sens, les humanités numériques (\acrshort{dh}) transforment les pratiques des sciences humaines et sociales en matière de production de savoir, tout en générant de nouvelles formes de connaissances. Elles dépassent les cadres disciplinaires traditionnels et s'instaurent comme une discipline autonome, partiellement définie se manifestant par une communauté de pratiques, incluant des chercheurs d'horizons divers dont certains sont issus de l'histoire de l'art. C'est aussi en cela que l'intronisation du numérique via les \textit{Digital humanities} induit une transformation épistémologique pour les disciplines qu'elles investissent de façon générale et pour l'histoire de l'art en particulier. Les \acrshort{dh} sont bien une trandiscipline. Ce changement transcende-t-il pour autant l'histoire de l'art ?

\subsubsection{(Re)définir l'histoire de l'art au prisme de l'informatique et de la géographie}
Dans ce contexte, on différencie la \textit{digitizing art history}\footcite{JOYEUX-PRUNELNumerique2021} -- qui décrit l'accès aux contenus numérisés -- de la \textit{computational art history}\footcite{BONFAITHumanites2021a} -- laquelle utilise l'informatique pour traiter, analyser et interpréter les matériaux historiques. Cette distinction n'est pas traduite par la communauté francophone qui utilise plutôt le terme de \enquote{histoire de l'art numérique}\footcite{DHIPARISRegistration2024} qui ne désigne pas pour autant les œuvres d'art créées numériquement ou nativement numériques. La littérature internationale et anglophone utilise aussi le terme de \textit{Digital art history}\footcite{IMPETTThere2022}. Cette diversité de termes reflète un manque de cohérence et témoigne du fait que la discipline alliant histoire de l'art et numérique ou informatique est effectivement en cours de redéfinition ou du moins en questionnement.

Cette communauté d'historiens et d'historiennes de l'art manifeste un fort intérêt pour les questions \enquote{d'analyse d'image, textuelle, de réseaux et spatiale}\footcite{BONFAITHumanites2021a} - également partagées au sein des \textit{Spatial humanities} (\acrshort{sh}). Les humanités numériques spatiales se regroupent en plusieurs groupes dont l'Alliance of Digital Humanities Organizations\footnote{Dont le site Web est à retrouver \href{https://adho.org/}{ici}} et le groupe Geohumanities\footnote{Dont le site Web est à retrouver \href{https://geohumanities.org/}{ici}} sont les plus représentatifs. Dans les champs singuliers de l'archéologie et de l'histoire, les \acrshort{sh} déploient de nouveaux outils tels que les systèmes d'information géographiques et historiques (\acrshort{sigh})\footcite{DUMENIEUsysteme2015} pour l'intégration des données historiques et patrimoniales à partir de plan anciens. Le projet Richelieu n'utilise pas de \acrshort{sig} historique mais un \acrshort{sig} strictement géographique auquel sont ajoutées des données temporelles comme nous l'avons vu dans la première partie du mémoire. Cette communauté qui se définit donc par ses pratiques \enquote{aborde aussi de nouvelles problématiques, comme l’enrichissement et la diffusion de référentiels géohistoriques ou patrimoniaux conformes aux standards et technologies du Web des  données.}\footcite{BRANDOIntroduction2021}, -- problématique que le projet Richelieu aborde en effet à l'exception du \textit{Linked data}.

L'intronisation de nouveaux outils de recherche et la diversification des pratiques conduit à l'élaboration de nouvelles méthodes de recherche. Par exemple, l'utilisation de la cartographie peut dévoiler \enquote{une image de la répartition spatiale des activités artistiques, littéraires et scientifiques et qui permet [...] d’installer une forme de compréhension de ces activités}\footcite{BESSEApproches2010} comme l'illustre très bien notre sujet de mémoire. A partir de cette étude de cas sur le projet Richelieu, on constate que les perspectives spatiales et numériques sont une force heuristique pour l'histoire de l'art\footcite{WINDHAGERReview}. 

Entraînent-ils pour autant redéfinition de la discipline ou élargissent-ils le champ des domaines de recherche en histoire de l'art? L'une ne serait-elle pas la question rhétorique de l'autre ? En effet, comment pouvons-nous encore parler d'histoire de l'art dans ce cadre intellectuel de recherche ? Comment relier la géohistoire à l'histoire de l'art numérique ? La \enquote{géohistoire de l'art numérique} ne serait-elle pas le mot-valise d'un domaine de recherche plutôt que d'une nouvelle discipline ?

Dans une récente revue qui contextualise la place du numérique dans l'histoire de l'art, trois figures éminentes de la discipline se demandent : \enquote{Si nous atteignons ce point critique de fusion transdisciplinaire, nous pourrions peut-être dire que nous sommes face à un nouveau domaine de recherche.}\footcite{JOYEUX-PRUNELComment2021}

Le cadre de ce mémoire ne permet pas d'examiner en profondeur cette question ni d'apporter des éléments de réponse élaborés. Toutefois, il apparaît clairement que l'interdisciplinarité mise en œuvre dans le projet Richelieu transforme les disciplines qu'il touche, notamment l'histoire de l'art. Pour certains chercheurs, cette transformation fait des champs de recherche en histoire de l'art une véritable transdiscipline en devenir\footcite{BONFAITHumanites2021a}. Même si l'\acrshort{inha}, en tant qu'acteur central en France, \enquote{ a le mérite de transformer et de démocratiser la discipline, notamment dans le domaine des humanités numériques}\footcite{HCERESRapport2024}, tout l'enjeu aujourd'hui semble résider dans le manque de lignes directrices à grande échelle. La communauté internationale qui se reconnaît par ses pratiques gagnerait en définition si elle se rassemblait en instaurant, par exemple, une icône institutionnelle ou une alliance organisationnelle qui \enquote{pourrait jouer en ce domaine le rôle moteur que joue la \acrshort{bnf} pour les bibliothèques}\footcite{DELMAS-GLASSHumanites2021}. Elle permettrait notamment d'écrire une historiographie qui questionne et pose des repères. Tel est l'enjeu épistémologique de l'histoire de l'art que soulève le projet Richelieu. 

\subsubsection{Conclusion du chapitre}
Ce dernier chapitre a permis de mettre en perspective, et surtout de discuter, plusieurs aspects clés du projet Richelieu, dont les domaines de recherche oscillent entre une communauté de pratiques et une discipline autonome en devenir qui gagnerait en visibilité à travers des médiations dans l'espace public. 

\subsubsection{Conclusion de la troisième partie}
En conclusion, cette dernière partie a présenté les perspectives offertes par le projet Richelieu. Les principes FAIR et de la science ouverte assure une forme de pérennité au projet pour l'inscrire au processus de patrimonialisation des plateformes numériques de recherche. Cet archivage viendrait aussi pérenniser les nouvelles pratiques de recherche d'un domaine de recherche en cours de définition. Puis, la géohistoire de l'art numérique gagnerait en visibilité à travers les médiations dans l'espace publics, renforçant ainsi son positionnement au sein des communautés de recherche. Finalement, le projet Richelieu s'inscrit dans une dynamique novatrice qui réaffirme son potentiel scientifique et culturel. 