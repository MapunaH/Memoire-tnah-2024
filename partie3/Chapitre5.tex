% PARTIE 3 - PERSPECTIVES %%%%%%%%
% CHAPITRE 5 %%%%%%%%%%%%%%%%%%%%%
%%%%%%%%%%%%%%%%%%%%%%%%%%%%%%%%%%
Ce chapitre ouvre la dernière partie du mémoire, qui aborde à la fois l'intégration de la carte en particulier et la pérennisation du projet dans son ensemble. Il s'organise autour de trois axes temporels distincts. Tout d'abord, nous analyserons la phase immédiate après le développement de la carte, en mettant l'accent sur son intégration au sein du cadre de travail (\textit{Framework}) de l'application. Ensuite, nous étudierons la durabilité du projet à moyen terme, en questionnant son accessibilité sur le Web et les bases qui la soutiennent, notamment les principes  \acrshort{fair} (\textit{Findable, Accessible, Interoperable, Reusable}). Enfin, nous nous pencherons sur la pérennité à long terme du projet de recherche, en accordant une attention particulière à l'archivage durable de ses données. 

%%%%%%%%%%%%%%%%%%%%%%%%%%%%%%%%%%%%%
% SECTION %%%%%%%%%%%%%%%%%%%%%%%%%%%
\section{Intégration immédiate : déploiement de la carte dans l'application}

\subsection{Principes élémentaires de Vue.js}
\subsubsection{Définition et avantages du \textit{Framework}}

Vue.js est un \textit{Framework} JavaScript progressif utilisé pour la construction d'interfaces utilisateur et d'applications Web, ce qui en fait un choix pertinent pour le projet Richelieu. Le \textit{Framework} permet de créer des interfaces interactives et dynamiques, et de faciliter le développement grâce à ses composants modulaires et réutilisables. Ces composants sont des blocs de code - analogues à des briques - comprenant du  \acrshort{html}, du  \acrshort{css} et du JavaScript, que l'on peut réutiliser dans l'ensemble de l'application pour la construire. Par exemple, une légende de la carte peut être développée sous forme de composant dans un fichier, puis appelée dans un autre fichier pour être affichée. Il est également possible de répliquer les composants dans d'autres projets gérés par une seule institution. C'est le cas de l'\acrshort{enc} qui reproduit plusieurs composants dans \href{https://dicotopo.cths.fr/}{DicoTopo}, \href{ https://adele.chartes.psl.eu/}{Adele}, \href{https://dev.chartes.psl.eu/ecco/}{Ecco}, \href{https://endp.chartes.psl.eu/}{e-NDP} ou encore le site des \href{https://theses.chartes.psl.eu/}{Thèses}\footnote{Cliquer sur le nom des projets pour accéder aux sites Web.}. Chaque composant est enregistré dans un monofichier (aussi appelé \textit{Single-File Components} pour \acrshort{sfc}) dont l'extension est \enquote{.vue}. Ce format est spécifique au \textit{Framework}. Ainsi les composants simplifient la gestion de l'interface et améliorent la maintenabilité du code.

De plus, Vue.js permet un développement progressif : tous les composants n'ont pas besoin d'être intégrés simultanément pour que le projet fonctionne. Il est possible d'en utiliser certains sans avoir à tout reconstruire depuis le début. Un autre atout majeur de Vue.js est son système de \enquote{ réactivité } : lorsqu'une donnée change d'état, l'interface utilisateur est mise à jour automatiquement, sans qu'il soit nécessaire de recharger la page sur le navigateur. Cela donne l'impression que toutes les données sont déjà présentes et que seule une action de l'utilisateur est requise pour les rendre visibles.

Grâce à ces caractéristiques, Vue.js s'avère être un cadre de travail adapté à des projets de tailles diverses, qu'il s'agisse de petits composants d'une page Web ou d'applications complètes comme le projet Richelieu. Enfin, il donne la possibilité à l'application d'évoluer facilement si celle-ci gagne en complexité ou en ampleur.

Cette définition des avantages du \textit{Framework} Vue.js donne quelques éléments de compréhension pour y intégrer la carte développée en silo jusque-là. 

\subsection{Schéma d'intégration}
Pour intégrer la carte développée dans un seul fichier  \acrshort{html} contenant du  \acrshort{css} et du JavaScript dans l'application Vue.js, il est nécessaire de découper les éléments du code en composants modulaires afin de respecter la structure du \textit{Framework}. Cette sous-section ne présentera pas le développement complet de tous les composants de la carte sinon le développement d'un seul à titre d'exemple. Nous choisissons de présenter l'intégration du fond de carte en .vue en utilisant Leaflet\footnote{Les explications de cette sous-section se sont appuyées sur la documentation du Framework Vue dont le site Web est disponible \href{https://fr.vuejs.org/api/sfc-script-setup}{ici}.}. 

Un  \acrshort{sfc} Vue permet de réunir la structure   \acrshort{html}, JavaScript et le style  \acrshort{css} d'un composant Vue dans un seul fichier .vue.  Celui-ci se compose de trois sections principales  : le \texttt{<script>} pour la logique JavaScript, le \texttt{<template>} pour le code  \acrshort{html}, le \texttt{<style>} pour les styles  \acrshort{css}.

\subsubsection{\texttt{<template>}}
Chaque fichier .vue peut contenir au maximum un bloc \texttt{<template>}. Ici, le bloc se compose de deux éléments : 
\begin{enumerate}
    \item \texttt{<div class="cartography-wrapper">} désigne l'élément qui contient la carte qu'on appelle communément un conteneur. Un conteneur peut contenir plusieurs \texttt{<div>} dont il est important de donner un nom d'identifiant pour que celui-ci soit ensuite repéré. Ici, il contient ainsi une \texttt{<div id="map-main">} désignant la carte. Il est d'usage de donner l'identifiant anglais \texttt{map} à un composant désignant une carte. La \texttt{div} est elle-même à considérer comme un conteneur dans lequel sera affichée la carte Leaflet. 
    \item \texttt{<div class="warn-wrapper">} est un bloc de code qui contient un message d'informations à destination des utilisateurs. Il s'agit d'un simple texte  \acrshort{html} encapsulé dans une balise de paragraphe \texttt{<p>}. 
\end{enumerate}
Remarquons que le terme \texttt{wrapper} désigne un code qui « enveloppe », ou englobe, d'autres composants. Dans notre cas, il enveloppe la carte \texttt{map-main} d'une part et le message d'avertissement \texttt{warn} d'autre part.

\begin{lstlisting}[language=HTML, caption=Code HTML du fond de carte pour la balise <template> en Vue.js]
<template>
  <div class="cartography-wrapper">
    <div id="map-main"></div>
  </div>
  <div class="warn-wrapper">
    <div class="warn">
      <p>Cette page est encore en développement</p>
    </div>
  </div>
</template>
\end{lstlisting}

\subsubsection{\texttt{<script>}}
Chaque fichier .vue peut contenir un seul bloc \texttt{<script>}, qui représente la logique JavaScript du composant en Vue.js. Il est important de distinguer \texttt{<script setup>} de \texttt{<script>} : le premier est recommandé lorsque l' \acrshort{api} Composition est utilisée, car il permet une meilleure organisation du code. En d'autres termes, le bloc \texttt{<script>} classique s'exécute une seule fois, au moment où le composant est importé pour la première fois. En revanche, le code dans \texttt{<script setup>} s'exécute chaque fois qu'une nouvelle instance du composant est créée. Le code actuel est structuré en deux parties : la première consiste en l'importation de divers éléments et fonctions, tandis que la seconde contient une fonction principale qui utilise ces imports.

\begin{lstlisting}[language=HTML, caption=Code HTML du fond de carte pour la balise <script> en Vue.js]
<script setup>
import { onMounted, ref } from "vue";
import L from "leaflet";
import { globalDefineMap } from "@utils/leafletUtils";
const map = ref();  // defined in onMounted

onMounted(() => {
  map.value = globalDefineMap("map-main");
  console.log(map.value);
})
</script>
\end{lstlisting}

\begin{enumerate}
    \item\texttt{import \{onMounted, ref\} from "vue";} est une ligne qui importe deux fonctionnalités clés de Vue.js.  \texttt{onMounted} est une fonction qui est exécutée une fois qu'un composant est chargé (c'est-à-dire après que le \acrshort{dom} soit prêt) et \texttt{ref} est utilisé pour déclarer des variables réactives dans Vue.js. 
    \item\texttt{import L from "leaflet";} est une ligne qui importe la bibliothèque Leaflet pour créer la carte interactive. La lettre \texttt{L} désigne un objet dans librairie Leaflet, il est plus communément compris comme l'objet Leaflet lui-même. 
    \item \texttt{import { globalDefineMap } from "@utils/leafletUtils";} est une ligne qui importe une fonction déjà développée dans un autre fichier .vue . Cet autre fichier réunit de nombreuses fonctions dont la création des tuiles cartographiques du serveur WMTS (Web Map Tile Service)\footnote{Le WMTS est un format utilisé pour afficher les cartes sous forme de tuiles, comme des images par défaut de 256 x 256 pixels qui sont ensuite combinées pour former une carte complète.}, surtout la fonction \texttt{globaleDefineMap}. Celle-ci initialise la carte Leaflet et inclut de nombreuses fonctionnalités dont le centre et les limites géographiques de la carte, les limites de zoom et les différentes couches tels que les fonds de cartes. 
    \item  \texttt{const map = ref();} déclare la variable pointant vers la carte, soit une référence réactive qui sera utilisée pour stocker la carte Leaflet après son intialisation à comprendre comme une instance.
\end{enumerate}

Après l'import, la fonction \texttt{onMounted} est appelée pour déclencher le chargement de la carte dans le \acrshort{dom}. Cela signifie que la carte Leaflet est initialisée en utilisant la fonction \texttt{globalDefineMap}, appliquée à l'élément avec l'identifiant \texttt{map-main}. L'objet carte ainsi créé est ensuite stocké dans la référence réactive \texttt{map.value}, permettant d'afficher les détails de la carte dans la console à l'aide de \texttt{console.log}. Ce dernier n'est pas strictement nécessaire, mais constitue une bonne pratique pour vérifier ce qui est affiché à l'écran. En résumé, la section \texttt{<script>} gère la phase de montage du composant, au cours de laquelle une instance de la carte est créée et associée à l'élément défini dans le \texttt{<template>}. Cette instance est ensuite stockée dans la variable réactive \texttt{map}. Cette fonction est un exemple pertinent de l'utilisation de Vue.js qui morcèle une grande fonction en plusieurs composants. 


\subsubsection{\texttt{<style>}}
Chaque fichier .vue peut contenir plusieurs blocs \texttt{<style>}. Vue.js propose également une option pour que le style  \acrshort{css} ne soit pas uniquement appliqué à ce composant avec la directive <style scoped>. Ainsi ce passage définit les styles appliqués aux éléments du composant. \texttt{cartography-wrapper} et \texttt{map-main} occupent toute la hauteur et la largeur de la fenêtre. Tandis que les éléments \texttt{warn-wrapper} et \texttt{warn} sont positionnés au centre de la fenêtre ce qui place le message d'avertissement au centre de l'écran. Enfin l'élément \texttt{warn} est définit comme un bloc sur fond blanc avec un contour dont l'opacité est suffisante pour assurer la lisibilité du texte. 
    
\begin{lstlisting}[language=HTML, caption=Code HTML du fond de carte pour la balise <style> en Vue.js]
<style scoped>
.cartography-viewer {
  height: calc(100vh - var(--cs-navbar-height));
  width: 100%;
}
#map-main {
  height: calc(100vh - var(--cs-navbar-height));
  width: 100%;
}
.warn-wrapper {
  position: absolute;
  transform: translateY(-100%);
  height: 100%;
  width: 100%;
  z-index: 999;
  display: grid;
  align-items: center;
  justify-content: center;
}
.warn {
  opacity: 1;
  background-color: white;
  border: var(--cs-border);
}
.warn > p {
  margin: 30px;
}
</style>
\end{lstlisting}


Il faut imaginer que les autres fonctionnalités de la carte s'ajoutent à ce composant de base. A terme, une page ou fichier .vue fait appel à une multitude de composants modulaires qui s'additionnent les uns aux autres. Développés indépendemment, les composants peuvent aussi être utilisés dans d'autres projets ou d'autres cartes de l'application. Il convient cependant de faire attention aux identifiants qui déterminent si tel ou tel effet s'applique à tel ou tel élément. 

Ce bref exemple d'intégration d'un élément  \acrshort{html} dans Vue.js démontre que le \textit{Framework} offre de vastes possibilités de développement. Il est réactif, progressif, modulaire et réplicable permettant ainsi aux développeurs de concevoir des interfaces entièrement personnalisées. Contrairement à certains autres outils, Vue.js ne propose pas de fonctionnalités prêtes à l'emploi comme Omeka S, ce qui en fait un atout pour ceux souhaitant développer à partir de zéro, \textit{from scratch}. 

\subsection{Réflexions à propos de Vue.js}

Vue.js peut aussi représenter une difficulté pour les développeurs moins expérimentés. Existe-t-il alors d'autres solutions applicatives capables de compenser ce manque d'expérience tout en répondant aux besoins du projet Richelieu ? 

\subsubsection{Omeka(s) versus Vue.js}
Omeka a été proposé comme alternative lors des discussions autour du projet. C'est un logiciel libre de gestion de bibliothèque numérique américain créée en 2008 et mis à disposition sous la licence \acrshort{gpl}. L'outil est développé par le Center for History and New Media (\acrshort{chnm}) de l'Université George Mason qui est aussi à l'origine du logiciel de gestion bibliographique Zotero\footcite{OMEKAAssociation}. Omeka est un système de gestion de contenu (\acrshort{cms}) conçu pour organiser, exposer et publier en ligne des données iconographiques, sonores et vidéo, accompagnées de leurs métadonnées. Grâce à sa conception modulaire, il permet d'adapter les fonctionnalités de chaque site via des \textit{plugins} et des thèmes. Il a été développé pour être principalement utilisé dans les musées, bibliothèques et archives\footcite{bibliotheque2019}.

La version sémantique d’Omeka, Omeka S, lancée en 2012, présente plusieurs avantages pour le projet Richelieu\footcite{bibliotheque2019}. Spécifiquement orienté vers le web de données, il facilite la gestion simultanée de plusieurs sites et favorise l’interopérabilité des données grâce à des standards tels que \acrshort{oaipmh}, Dublin Core, \acrshort{jsonld}, \acrshort{rdf}, \acrshort{iiif}, et des  \acrshort{api} REST. Ainsi, Omeka S est conforme aux recommandations du \acrshort{w3c}. Son installation s’effectue sur un environnement Linux avec un serveur Apache et une base de données MySQL. Sa polyvalence, sa dynamique d’utilisation et sa facilité d’installation sont autant d’atouts pour le projet.

Omeka S permet également d'associer des textes à des images sous forme de mosaïque ou d'affichage cartographique. Cela correspond à certains besoins du projet Richelieu, notamment pour l'indexation de thématiques hiérarchisées, la création de réseaux de relations entre sources, et la présentation visuelle sous forme de catalogue ou de cartes. Pour la partie cartographique, Omeka S permet d'intégrer des concepts géographiques via l' \acrshort{api} GeoNames, de créer des couches d'information avec \acrshort{qgis}, de générer des fichiers \acrshort{geojson} contenant des données géographiques (comme des polygones), et de mapper ces informations sur des cartes interactives avec Leaflet, intégrées directement dans la plateforme. De nombreux projets de recherche se utilisent ce logiciel pour leur recherche. Parmi les plus connus, on peut citer les projets de l' \acrshort{inha}, tels que le \href{https://digitalmuret.inha.fr/s/accueil-muret/page/accueil}{Digital Muret} et \href{https://sismo.inha.fr/s/fr/page/welcome}{Sismographie des luttes}. Concernant l'utilisation de la cartographie, le site de Omeka référence, par exemple, les projets  \href{https://genovefa.bsg.univ-paris3.fr/s/voyages-savants/page/lignes-d-horizon}{Voyages Savants} et \href{https://imageo.univ-lorraine.fr/s/imageo/page/carte}{Imageo}\footnote{Cliquer sur le nom des projets pour accéder aux sites Web.}.  

Pourquoi alors avoir opté pour une application Flask en \textit{back-end} et Vue.js en \textit{front-end}, plutôt qu'Omeka S ?

\subsubsection{Des raisons contextuelles}

Le choix d'une application Flask en \textit{back-end} et Vue.js en \textit{front-end} plutôt qu'Omeka S peut être justifié par plusieurs facteurs liés à la flexibilité, la personnalisation et les besoins spécifiques du projet Richelieu. Bien qu'Omeka S offre une solution robuste et modulaire, adaptée à la gestion de contenus culturels et scientifiques, son cadre reste relativement structuré et limité aux fonctionnalités prédéfinies par ses plugins.

Vue.js, associé à Flask, permet une plus grande liberté dans la conception d'une application web entièrement sur mesure. Cette combinaison permet de développer des fonctionnalités spécifiques au projet, d'optimiser la performance, et de mieux intégrer des flux de travail interactifs, en particulier lorsque des interfaces dynamiques et réactives sont requises. Vue.js, bien que demandant une certaine expertise, donne la possibilité de construire des interfaces utilisateur complexes tout en maintenant une architecture frontale légère et flexible.

En conclusion, bien que des développeurs moins expérimentés, comme l'auteure, aient rencontré des difficultés à maîtriser Vue.js, et, une fois que l'ingénieur du projet s'est familiarisé avec la courbe d'apprentissage de cette technologie, l'application s'est révélée plus performante et interactive, répondant ainsi mieux aux exigences du projet. Cette solution offre un contrôle plus complet sur le développement, ainsi qu'une meilleure extensibilité et adaptabilité à long terme, ce qui la rend plus appropriée qu'Omeka S pour ce type de projet.

%%%%%%%%%%%%%%%%%%%%%%%%%%%%%%%%%%%%%
% SECTION %%%%%%%%%%%%%%%%%%%%%%%%%%%
\section{Durabilité à moyen terme : principes  \acrshort{fair} et science ouverte}\label{section:intero}

Une fois que la carte est développée et intégrée à l'application Web, la plateforme est complète et accessible en ligne (voir \ref{section:web}). Cependant, dans un environnement numérique toujours plus dense, retrouver ce projet parmi la multitude de pages interconnectées peut s'avérer complexe sans des repères clairs permettant de l'identifier et de le localiser. Dans le domaine des sciences et de la recherche académique, les principes  \acrshort{fair} jouent ce rôle de guide sur le Web. Pour le projet Richelieu, ces principes forment un écosystème propice à sa découvrabilité et à sa durabilité à moyen terme, assurant non seulement l'accessibilité des données, mais aussi leur interopérabilité et leur réutilisation future. 

\subsection{Les principes \acrshort{fair}}

\subsubsection{Définition générale}
Initialement conçus pour contrer les effets de cloisonnement générés par les projets, les principes  \acrshort{fair} reposent sur quatre notions clés.\footcite{WILKINSONInteroperability2017} Premièrement, les données doivent être faciles à trouver (\textit{Findable}), c'est-à-dire identifiées par des identifiants uniques, persistants et résolvables, et accompagnées d'informations contextuelles exploitables par des machines, permettant ainsi leur indexation et leur découverte. Deuxièmement, elles doivent être accessibles (\textit{Accessible}) à la fois par les humains et les machines via des protocoles de communication clairement définis, incluant si nécessaire des règles d'autorisation ou d'authentification. Troisièmement, les données doivent être interopérables (\textit{Interoperable}), ce qui signifie qu'elles peuvent être utilisées dans différents systèmes grâce à des vocabulaires contrôlés et ontologies partagées, dans des formats et standards ouverts compréhensibles par les machines. Enfin, elles doivent être réutilisables (\textit{Reusable}), grâce à des descriptions claires sur leurs informations contextuelles et de provenance, et liées à des licences autorisant leur exploitation et intégration avec d'autres sources de données similaires, tout en étant correctement citées dans des nouveaux contextes scientifiques ou culturels. L'application des principes  \acrshort{fair} dans les projets de recherche favorise une ouverture et un partage des informations à long terme, en facilitant leur interconnexion avec d'autres bases de données ou plateformes. Qu'en est-il pour le projet Richelieu ? 

\subsubsection{Appliqués au projet Richelieu}

\paragraph{Une bibliothèque \acrshort{iiif}}
Le projet Richelieu adhère à plusieurs de ces principes notamment avec l'utilisation du protocole interopérable \acrshort{iiif} (voir le passage à ce propos :\ref{sous-section:web}). Afin de faciliter la découverte des données iconographiques du projet (\textit{Findable}), chaque image est accompagnée d'un manifeste \acrshort{iiif}. Il s'agit d'un fichier unique au format standard du Web JSON-LD, qui contient les informations essentielles sur la structure de l'image numérique. Cette découvrabilité est également renforcée par  la réutilisation (\textit{Reusable}) et l'accessibilité (\textit{Accessible}) des images : le manifeste peut être utilisé par des machines pour afficher l'image dans une visionneuse, tout en étant lisible par l'homme grâce aux métadonnées qu'il contient. En effet, le manifeste inclut des éléments comme le titre de l'image, l'auteur, les dimensions mais surtout l'identifiant unique, les licences, et les conditions de réutilisation. Chaque source iconographique est donc liée à un identifiant unique conforme à la structure \acrshort{ark} (voir le passage à ce propos\ref{}), garantissant un accès durable aux ressources, tant pour le projet Richelieu que pour les institutions productrices. 
Appliquer le standard \acrshort{iiif} offre donc de nombreux atouts pour rendre visible un projet de recherche numérique en histoire de l'art. C'est \enquote{une façon inédite de décloisonner les collections, à laquelle les licences libres et ouvertes apportent la brique juridique, et le standard \acrshort{iiif} la brique technologique} \footcite{DELMAS-GLASSHumanites2021}. Plus largement, les principes  \acrshort{fair} augmentent la capacité des autres chercheurs ou institutions à exploiter les données, renforçant ainsi leur durabilité si elles sont en effet réutilisées. Comme c'est le cas du projet Richelieu qui partage un protocole déjà utilisé par d'autres institutions faisant remarquer au passage l'approche fructueuse des principes \acrshort{fair}. Par conséquent, un projet est non seulement pérennisé, mais il contribue également à enrichir le réseau global de la recherche et des collections culturelles. 

Toutefois, le protocole \acrshort{iiif} n'est pas seul à favoriser la durabilité du projet, il existe aussi d'autres possibilités comme le souligne Delms-Glass : \enquote{les principes du Web des données et les protocoles \acrshort{iiif} sont en effet deux piliers fondamentaux dans la stratégie d’ouverture et de partage des données numériques des \acrshort{glam}}\footcite{DELMAS-GLASSHumanites2021}. 

\paragraph{Le Web de données}
Une autre manière d'implémenter les principes \acrshort{fair} consiste en effet à utiliser le Web de données, également appelé Web sémantique. Proposé par Tim Berners-Lee, l'inventeur du Web, cette approche, théorisée entre 1998 et 2006\footcite{BERNERS-LEELinked2009}, vise à créer des liens entre les données pour permettre aux machines et aux êtres humains d'explorer le réseau des informations interconnectées. Fonctionnant comme un langage, le Web de données utilise des ontologies, qui servent de grammaire et de vocabulaire pour communiquer\footcite{BERMEStechnologies2013}. Toutefois, le projet Richelieu n'a pas opté pour cette intégration, et cela pour des raisons que nous avons évoquées (voir le chapitre \ref{}). Il n'utilise pas d'ontologies ni de base de données en graphes. Il s'avère que la mise en place du Web de données dans les projets de recherche en histoire de l'art rencontre quelques difficultés comme le rappelle Delmas-Glass dans le cadre des institutions patrimoniale : \enquote{là où le standard \acrshort{iiif} peut être adopté en quelques mois ou moins, rendre ses données pleinement compatibles avec le Web des données peut être un chantier pluriannuel en fonction de l’état des lieux.}\footcite{DELMAS-GLASSHumanites2021} Cette contrainte temporelle vient justifier les choix privilégiés par le projet Richelieu pour des solutions immédiatement opérationnelles, comme le \acrshort{iiif}, qui répondent déjà aux besoins d'interopérabilité et de partage des données dans des délais plus raisonnables.

Notons que l'application des principes \acrshort{fair} participe à la dynamique de la science ouverte dont le projet Richelieu en fait aussi la promotion. 

\subsection{La science ouverte}
En novembre 2023, le projet Richelieu s'est vu en effet desservir le prix de la science ouverte des données de la recherche par le comité éponyme qui assure la mise en oeuvre de la politique nationale de la science ouverte\footcite{KERVEGANPrix2023, GTSODONNEESDECOUPERINReutiliser2024}. Composé de plusieurs instances, il mobilise les acteurs de l'enseignement et de la recherche pour accompagner dynamiquement la stratégie de la science ouverte\footcite{Presentation2024}. Place sous double tutelle du ministère de l'Enseignement supérieur et de la Recherche et du ministère de la Culture, l' \acrshort{inha} y participe activement. Il a notamment publié en 2023, une charte de la science ouverte\footcite{INHACharte2024} pour réaffirmer son engagement en faveur de l'ouverture et du partage des résultats et données de la recherche en histoire de l'art, conformément aux stratégies du Plan national pour la science ouverte (\acrshort{pnso})\footcite{INHAVers2023}. L'objectif est de rendre la recherche plus accessible, en facilitant la diffusion des publications et des données scientifiques à un public élargi. L' \acrshort{inha} joue un rôle actif dans la mise en œuvre des principes  \acrshort{fair} (Facile à trouver, Accessible, Interopérable, Réutilisable), qui s'intègrent dans cette dynamique, en garantissant que les données soient non seulement ouvertes, mais aussi bien structurées et exploitables sur le long terme. Ces principes contribuent à un écosystème durable où les informations scientifiques restent disponibles et interconnectées, tout en respectant les exigences éthiques et légales de la recherche.

Toutefois, il est important de ne pas confondre les principes  \acrshort{fair} avec ceux de l'\textit{Open Data}, de l'\textit{Open Access} et de l'\textit{Open content}, qui relèvent de la Science ouverte. Même si les principes \acrshort{fair} sont souvent couplés avec les notions de la Science ouverte, dans la mesure où des données accessibles et réutilisables doivent être en accès libre. Ainsi les principes \acrshort{fair} concernent avant tout la gestion et la qualité des données pour en faciliter la découverte et l'utilisation, sans impliquer nécessairement qu'elles soient librement accessibles. Ainsi, une donnée peut être  \acrshort{fair} tout en restant soumise à certaines restrictions d'accès : elle est bien décrite et organisée ce qui peut la rendre accessible par des machines mais n'est pas forcément gratuite ou ouverte au public pour des raisons éthiques, légales ou liées à la confidentialité. Cette distinction est particulièrement importante dans la recherche en histoire de l'art, où certaines données peuvent concerner des œuvres protégées par le droit d'auteur ou des artefacts nécessitant une gestion spécifique.

\subsubsection{Quelques freins inhérents au domaine d'application}
En effet l'application des principes  \acrshort{fair} dans le domaine de la recherche, notamment en histoire de l’art, se heurte à plusieurs freins inhérents à la nature du domaine. Un des principaux obstacles concerne la structuration et la publication des métadonnées, qui sont souvent complexes à standardiser et à relier aux autres projets. Le liage des données via un vocabulaire commun est crucial pour permettre une interconnexion avec le Web de données et participer pleinement à l’écosystème de la Science ouverte, mais cela nécessite des efforts importants en termes de normalisation et d’interopérabilité comme nous l'avon vu au début du présent chapitre. Par ailleurs, des freins traditionnels subsistent, notamment la crainte de perdre le contrôle sur les données ou de voir celles-ci détournées. 

Les enjeux juridiques, comme les droits des ayants droit, sont également un frein majeur à l’ouverture des données. Certains acteurs hésitent ou refusent catégoriquement de publier des données ou des images, particulièrement en raison des exigences de protection des droits d’auteur ou des accords avec les détenteurs des droits. C'est par exemple le cas pour les oeuvres de l'artiste Pablo Picasso dont Picasso Admistration gère la totalité des droits, y compris dans le cadre de fouilles de donnée\footnote{La première et très récente page du site de Picasso Administration, https://www.picasso.fr/, indique que celle-ci exerce son droit d'opposition pour qu'aucune des oeuvres de l'artiste ne puisse être utilisée dans la fouille de donnée sans son autorisation préalable.}. Sur le plan technique et humain, un manque de formation et le coût élevé de l’intégration des principes  \acrshort{fair} peuvent décourager certains projets de s’engager dans cette voie. De plus, certaines réticences culturelles subsistent, comme la méfiance vis-à-vis des normes eurocentrées. C'est notamment le cas des collections liées aux communautés autochtones, où les principes \acrshort{care} (avantages Collectifs, Autorité, Responsabilité, Éthique) publiés par la Global Indigenous Data Alliance en 2019, soulignent la nécessité de respecter la souveraineté de leurs données afin d'éviter l’universalisation de l’accès aux données, qui peut entrer en contradiction avec les épistémologies des communautés sources\footcite{REDDENSouverainete2023}.

Ces freins techniques, juridiques et culturels montrent qu’au-delà des aspects technologiques, l’application des principes  \acrshort{fair} dans l'écosystème de la science ouverte doit prendre en compte une diversité d’enjeux socio-culturels et éthiques\footcite{WILKINSONInteroperability2017}, tout en respectant les spécificités de chaque projet et communauté.

Pour conclure, nous constatons que la durabilité à moyen terme du projet Richelieu repose sur l’application des principes fondamentaux du Web dans le domaine académique qui garantissent une structure solide pour assurer la pérennité des données, en favorisant leur conservation, partage et réutilisation dans le temps. Ils sont essentiels pour garantir que les données restent pertinentes malgré les évolutions technologiques et institutionnelles. Bien qu'il ne s'agisse pas de règles strictes, ces principes fournissent des lignes directrices claires qui aident à la gestion rigoureuse des données numériques, contribuant à enrichir le réseau global de la recherche et des collections culturelles.  Enfin, malgré ces défis, leur adoption dans le cadre du projet Richelieu renforce son archivage et sa longévité pour les générations futures.

%%%%%%%%%%%%%%%%%%%%%%%%%%%%%%%%%%%%%
% SECTION %%%%%%%%%%%%%%%%%%%%%%%%%%%
\section{Pérennité à long terme : archivage et conservation des données}
Archiver le projet Richelieu a pour objectif d'assurer la pérennité à long terme de ses données, de ses résultats et de ses ressources produites, telle que la carte.  Mais archiver un projet de recherche en sciences humaines et sociales présente plusieurs enjeux liés à la nature des données. De quel type d'archivage s'agit-il pour le projet Richelieu ? Quelles sont les données concernées par l'archivage pérenne ? Quelles sont les préconisations pour archiver une plateforme numérique du Web ? 

Il est d'usage que les institutions émettrices et productrices des données de recherche appliquent une politique d'archivage pérenne. Dans le domaine de l'information numérique, cette politique consiste à conserver et à \enquote{maintenir accessibles et intelligibles les archives (documents et données) au cours du temps, que ce soit pour une durée courte ou longue voire très longue dans le cas des archives définitives}\footcite{FRANCEARCHIVESperennisation2023} et ce malgré les défis liés à la donnée codée, à la dépendance technologique, et à la fragilité des supports.  Pour ce faire, cela nécessite des moyens techniques et organisationnels qui tient compte aussi de l'obsolescence des matériels et des formats, ainsi que de l'intégrité des informations. 

L' \acrshort{inha} ne dispose pas encore de cette politique en interne mais c'est un des principaux objectif de son comité de pilotage de la Science Ouverte. Pour le moment, il dispose d'une stratégie d'hébergement à long terme de ses projets de recherche. En effet, même si de nombreux projets se sont clôturés au fil du temps\footnote{Digital Muret, 2017-2023 ; Sismo, terminé en 2021 ; OMCI ; Webdocumentaire du portail de Conques, avant 2018}, ils restent en ligne car le SNR ainsi que le système informatique de l'institut assurent leur maintenance.  

Qu'en sera-t-il pour le projet Richelieu et pour la phase de RICH.DATA II? A l'heure où le présent mémoire est écrit, il a été convenu que l' \acrshort{inha} assure la maintenance de la plateforme car il conserve l'hébergement du site Web et de la base de données. A terme, il conviendrait peut-être que le projet soit hebergé sur la plateforme TGIR Huma-Num comme c'est déjà le cas pour de nombreuses plateformes de recherche. Nakala, un entrepôt des données en sciences humaines et sociales est aussi une perspective pour déposer les jeux de données les plus lourds.  

Que conviendra-t-il d'archiver ? Le site Web et ses différentes pages sont-ils des archives? Le code écrit en différents langages, comme le JavaScript qui propose une multitude d'interactions, peut-il seulement être archivé, totalement ou partiellement ? Surtout, l'archivage pose une question de stockage : où archiver ? Nous ne nous occuperons pas de poser la question des formats et des normes d'archivage du Web et des données numériques sinon de présenter les préconisations d'archivage. 

Assurer un archivage pérenne au projet Richelieu relève \textit{a priori} de deux entités juridiques et de deux techniques d'archivages différentes. D'une part, il y a l'archivage des données du projet, qu'elles soient scientifiques ou administratives, et, d'autre part, il y a l'archivage du site Web. 

\subsection{Verser les données administratives et scientifiques}

\subsubsection{Des préconisations}
Un référentiel de gestion et de traitement des archives de la recherche synthétise les règles d'usages et de conservation des archives des laboratoires de recherche, des chercheurs dans les universités et des organismes de recherche.  Rédigé en 2016 par le groupe Aurore\footnote{La section Aurore de l'\acrshort{aaf} désigne un groupe d'archivistes des universités, rectorats, organismes de recherche et mouvement étudiants.} de l'Association des archivistes français (\acrshort{aaf}), il précise que les producteurs étant divers, la nature des documents à archiver l'est également : il peut s'agir d'archives papier ou numérique \enquote{produites et reçues par un chercheur, une équipe de chercheurs, l'administration d'un laboratoire ou d'un département scientifique}\footcite{AUROREReferentiel2016}. Le référentiel distingue plusieurs catégories d'archives de la recherche : celles liées à la gestion et à la direction, celles relatives aux travaux de recherche (incluant les programmes, la documentation, la production scientifique ainsi que la diffusion des résultats), ainsi que celles concernant la valorisation, l'enseignement et la formation. Pour chaque typologie documentaire, le référentiel préconise des indicateurs pour la durée d'utilité administrative (\acrshort{dua}), le sort final appliqué aux documents (conservation définitive et intégrale, tri avant versement et conservation, ou destruction intégrale et définitive) et quelques observations complémentaires. 

\subsubsection{Les archives du projet}
Selon ce référentiel, le projet Richelieu est concerné pour trois types de ses archives. Tout d'abord, il s'agit des archives administratives liées à la gestion, aux ressources humaines et à la comptabilité tels que les rapports de projets, les fiches de poste, les ordres du jour et compte-rendus de réunion. Puis, les archives de valorisation de la recherche tels que les contrats avec les partenaires, la participation à des congrès, colloques, journées d'études ou symposium sont concernées. Enfin, et surtout, sont intégrées les archives des travaux de recherche qui réunissent à la fois les dossiers du programme décrivant le projet, et surtout les collectes et traitements de données comme les \enquote{bases de données brutes rassemblées dans de grandes bases de données et plateformes de recherche}\footcite{AUROREReferentiel2016}. Nous nous intéressons moins à l'archivage des données administratives qu'à l'archivage des données issues des travaux de la recherche scientifique. 

Il convient de souligner que les données scientifiques jouent en effet un rôle essentiel à plusieurs égards, notamment pour retracer l'évolution d'un projet de recherche. Par exemple, il est possible de retracer l'historique du projet en comparant les plateformes et bases de données aux programmes qui définissent les objectifs. Cela aide à comprendre comment le projet a évolué et pourquoi certaines technologies ont été préférées à d'autres, en justifiant ainsi les choix stratégiques et méthodologiques au fil du temps.

Pour ces dernières, la durée de conservation préconisée par l'\acrshort{aaf} varie en fonction de l'objet de la recherche : elle peut durer tant que dure le projet et s'étire jusqu'à 10 après la fin de celui-ci. Les base de données initialement conçues par les chercheurs, sur tableur, et la base de données finale sur Postgres sont donc des documents numériques à conserver. Force est de constater que l'archivage des données administratives et scientifiques du projet Richelieu suiverait un processus plutôt classique recommandé par le référentiel mais qui n'inclue pas explicitement le site Web et l'outil cartographique. Ces derniers pourraient être pérennisés via la publication de \textit{datapapers}.

\subsection{Signaler les sites Web}

\subsubsection{Méthode et raisons}
L'archivage pérenne des projets et plateformes publiées sur le Web sous forme de site Web est assuré par le dépôt légal du Web de la \acrshort{bnf}\footcite{Depot}. Bien que des robots sont utilisés pour moissonner automatiquement une grande partie des sites du Web français, ce processus n'est pas exhaustif. Non seulement un tri est effectué lors de la collecte mais surtout certains contenus peuvent ne pas être détectés ou être protégés par des restrictions techniques, comme l'utilisation de bases de données ou des accès payants. C'est pourquoi les propriétaires ou gestionnaires de sites Web, notamment dans le cadre de projets de recherche, sont encouragés à signaler manuellement leurs sites à la \acrshort{bnf} afin de garantir une prise en compte dans les collectes futures. 

Le signalement des sites Web à la \acrshort{bnf} peut se faire simplement en soumettant une \acrshort{url} via un formulaire dédié\footcite{Signaler}. En plus de l'\acrshort{url}, il est possible de fournir des informations supplémentaires telles que le thème du site ou son importance scientifique, ce qui aide la \acrshort{bnf} à prioriser et à contextualiser les sites lors de l'archivage. Ce processus est particulièrement pertinent pour les plateformes de recherche, qui évoluent souvent rapidement et nécessitent une attention particulière pour assurer que leurs résultats et données soient préservés à long terme. Le dépôt légal du Web permet ainsi de documenter ces ressources dans un cadre institutionnel reconnu.

Enfin, ce signalement permet de favoriser l'intégration de ces sites dans une politique d'archivage plus large, en garantissant leur accessibilité à travers les décennies, tant pour les chercheurs actuels que pour les générations futures. Certains chercheurs à venir pourraient s'intéresser à l'évolution de l'histoire de l'art en comparant diverses plateformes de recherche. Ainsi l'archivage du Web, en collaboration avec la \acrshort{bnf} et d'autres institutions, comme l'\acrshort{ina} pour les contenus audiovisuels et les \acrshort{an} pour les données administratives, contribue à la constitution d'un patrimoine numérique riche et accessible. Les projets de recherche, souvent basés sur des plateformes numériques interactives, peuvent profiter de cette opportunité pour assurer la pérennité de leurs contenus et la transmission de leurs résultats. 

\subsubsection{Quelques limites}

Toutefois l'archivage des plateformes interactives, comme les projets de recherche en ligne ou les sites de visualisation de données, présente des limites importantes. L'un des principaux défis est l'impossibilité de capturer les interactions dynamiques des utilisateurs avec les données, telles que les explorations, les clics, et les échanges en temps réel. Contrairement aux contenus statiques, ces éléments interactifs échappent aux systèmes actuels d'archivage, ce qui entraîne une perte d'une partie de l'expérience utilisateur et de l'évolution de la plateforme. En outre, l'absence de mécanismes d'émulation, permettant de recréer l'environnement d'origine, constitue un frein majeur à la conservation des plateformes complexes dans leur forme initiale.  Lorsqu'une plateforme interactive est archivée sur une autre infrastructure, comme un dépôt numérique, le rapport de l'utilisateur aux données change. Les interactions contextuelles ne peuvent pas être préservées mais est-ce seulement souhaitable voire possible ?\footcite{BERMESEcran2024} Une étude détaillée apporterait davantage d'éléments de réponse mais ces développements sont en dehors de notre sujet d'étude. 

     Dans ce contexte, l'archivage limite en partie la compréhension future du projet et de son impact dans son environnement interactif initial. Il permet toutefois de laisser une trace de ces initiatives numériques en histoire de l'art. 

\subsection{D'autres initiatives exemplaires}

En complément des archivages institutionnels, des initiatives isolées émergent, offrant des solutions pour l'archivage du code des applications. Par exemple, Software Heritage\footnote{Dont le site Web est accessible \href{https://www.softwareheritage.org/}{ici}.}, marque déposée par l'\acrshort{inria} en 2016, vise à collecter, organiser, préserver et rendre accessible à tous le code source de logiciels\footcite{DICOSMO20442022}. Ce dépôt numérique est disponible sur la plateforme HAL depuis 2019\footcite{BADOLATOlogiciel2022}, permettant aux chercheurs et développeurs non seulement de déposer le code source de leurs logiciels, mais aussi de rechercher, citer et modifier des logiciels existants. Les \enquote{logiciels de recherche peuvent se formaliser de différentes façons (une plateforme, un intergiciel, un \textit{workflow} ou une bibliothèque, module ou greffon d’un autre logiciel) et être ainsi en interaction dans un écosystème ou au contraire plus autonomes}\footcite{BADOLATOlogiciel2022}. 
La plateforme du projet Richelieu rentre-t-elle dans cette catégorie ? Pour le moment le code source de l'application est déposé de façon stable sur \acrshort{gitlab}, reconnu par le ministère de la Culture, sous licence \acrshort{gnu} \acrshort{gpl} 3.0\footcite{INHAVers2023}. Mais c'est un dépôt temporaire. Il serait intéressant de poser la question directement aux instances responsables pour en être sûr. Toujours est-il que cette initiative de Software Heritage est exemplaire pour archiver le code source des applications, ce qui faciliterait la constitution d'un catalogue unifié des plateformes de recherche en histoire de l'art, contribuant ainsi à une meilleure compréhension du paysage numérique de ce domaine.

En conclusion, rappelons qu'en aucun cas qu'il ne s'agit de prescriptions règlementaires sinon de recommandations des bonnes pratiques d'archivage. Enfin, archiver le projet Richelieu garantit sa lisibilité et intelligibilité à long terme, tout en contribuant au processus de patrimonialisation du numérique. Ce type d'archivage permet non seulement de préserver les données et ressources associées, mais aussi de valoriser le projet en tant qu'élément du patrimoine scientifique et culturel numérique.


\subsubsection{Conclusion du chapitre}
Cet avant-dernier chapitre expose les premières perspectives du projet Richelieu d'une part et de la carte d'autre part. Même si des défis sont inhérents à l'histoire de l'art, la pérennité de la plateforme repose sur des piliers de la science ouverte. Ce positionnement contribue à la redéfinition de la discipline. 