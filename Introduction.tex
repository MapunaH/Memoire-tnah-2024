Dans le film \textit{Retour vers le futur}\footcite{ZEMECKISRetour1985}, les personnages voyagent dans le passé à l'aide d'une machine à remonter le temps, mais leur expérience tourne mal. De nombreuses œuvres, tant littéraires \footcite{BARJAVELVoyageur1943, ASIMOVFin1955, BAXTERVaisseaux1996, WELLSmachine1895} que cinématographiques \footcite{MARKERJetee1962, PALMachine1960, DIVERSDoctor1963, POIREVisiteurs1993, CURTIStime2013, NOLANTenet2020}, explorent le thème du voyage dans le temps, qu'il s'agisse de revisiter le passé ou d'explorer le futur. Cependant, cette idée ne se limite pas à la fiction artistique ; elle peut également revêtir une dimension scientifique et historique. C'est précisément l'objectif du projet \enquote{Richelieu. Histoire du quartier}, qui vise à revisiter le passé d'un quartier parisien situé entre le Palais-Royal, l'Opéra Garnier et la place des Victoires, sur une période allant de la fin du XVIIIe siècle au début du XXe siècle. Ce lieu, désigné comme le \enquote{quartier Richelieu}, est examiné sous les angles architecturaux, culturels, économiques et sociaux. Ce projet, initié en 2018 par un consortium d'institutions patrimoniales, dont l'\acrlong{inha} (INHA) et la \acrlong{bnf} (BnF), a pour objectif de créer une plateforme numérique, accessible via une interface Web. Celle-ci permettra aux chercheurs et au grand public de découvrir ce quartier à travers des sources documentaires numérisées, des parcours historiques centrés sur les lieux de mémoire, et des articles thématiques rédigés pour enrichir la compréhension de ce patrimoine ancré au cœur des lieux du savoir et des arts. Cette démarche est comparable à une nouvelle forme de \textit{retour vers le futur}, où les technologies numériques, et notamment la cartographie Web, offrent une nouvelle manière d'explorer les vestiges du passé.

Le projet Richelieu s'inscrit à la croisée de plusieurs disciplines : histoire, histoire de l'art, histoire de l'architecture, géographie, informatique, géomatique et humanités numériques, entre autres. Ces disciplines utilisent des outils spécifiques, dont la visualisation des données, définie comme une \enquote{présentation visuelle sur un écran, sous forme d'image alphanumérique ou graphique, d'un ensemble d'informations traitées par des moyens informatiques} (Larousse). Dans le cadre du projet Richelieu, ces informations sont historiques et géographiques : les dates sont associées à des coordonnées géographiques. Ces informations, quant à elles, prennent la forme d'une carte : \enquote{une représentation conventionnelle et plane, de phénomènes concrets ou abstraits, toujours localisables dans l'espace} (Larousse). Cette cartographie est exposée sur le Web : les données géohistoriques sont reliées au système d'hypermedia offrant un accès aux ressources en ligne. Ce processus innovant transforme la manière dont le public interagit avec le patrimoine historique.

Cependant, la représentation graphique de données géographiques n'est pas une innovation récente. Des exemples de visualisations de données, bien que rudimentaires, existaient déjà dans les peintures rupestres qui représentent des chaînes de montagne comme c'est le cas de la carte retrouvée sur les parois de la grotte de Belinda en Italie du Nord datant de 2000 ans avant notre ère\footcite{LAMBERTBreve2016}. Avec l'avènement des mathématiques modernes, la cartographie s'est considérablement développée, comme en témoignent les cartes célestes. Parmi les réalisations emblématiques, souvent citée comme la première visualisation de données, figure la \enquote{Carte figurative des pertes successives en hommes de l’armée française dans la campagne de Russie 1812–1813} créée par Charles Joseph Minard en 1869. Un siècle plus tard, les cartes se diversifient et sont créées par ordinateur, la première révolution arrive avec la mise en place du premier système d'information géographique (\acrshort{sig}) au Canada en 1960 - quelques années avant que la Terre ne soit photographiée depuis l'espace en 1966. Cette représentation géographique subit un nouveau bouleversement quand la multinationale Google propose au monde entier des cartes et plans alors qu'il s'agissait jusque là d'un domaine réservé aux Etats-nations.

Dès lors, la question se pose : quelles sont les possibilités offertes par la cartographie Web pour visualiser les sources historiques et géographiques dans le cadre du projet Richelieu ? Comment peut-elle représenter conjointement le temps et l’espace ? Quelles technologies sont disponibles pour cette représentation géohistorique ? Comment intégrer la carte dans une vision à long terme afin d'assurer la durabilité de ces représentations ? Et enfin, qui sont les producteurs et les utilisateurs de cette cartographie Web?

Ce mémoire n’a pas pour ambition d’apporter des innovations en matière de cartographie Web, mais se propose d'analyser le projet Richelieu comme cas d’étude. Structuré de manière chronologico-thématique, le plan débute par une présentation du contexte historique et numérique du projet, ainsi que des acteurs et des corpus impliqués. Une analyse de l'état de l'art de la cartographie Web permet ensuite de poser les bases méthodologiques pour le développement de la carte. Enfin, le mémoire explore les différents prototypes réalisés jusqu'à présent. Le mémoire est à rendre avant que le stage ne se termine,  ce qui implique que certaines propositions resteront inachevées, à l'état de chantier. Enfin, il conclut en questionnant l'impact du projet Richelieu sur les disciplines concernées, ainsi que sur les moyens d'assurer sa pérennité et sa bonne réception auprès des publics.