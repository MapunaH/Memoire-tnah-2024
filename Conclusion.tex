Dans ce mémoire, nous avons examiné le projet \enquote{Richelieu.Histoire du quartier}, afin de déterminer en quoi l'utilisation de la cartographie Web en histoire de l'art constitue un \textit{nouveau} retour vers le futur. La question est aussi ambitieuse que le projet qui souhaite reconstituer l'histoire du quartier Richelieu sur deux siècles grâce à l'utilisation des technologies du Web. 

Pour conclure, le premier chapitre expose le contexte historique de production du projet, en soulignant la diversité des sources et les objectifs principaux, à savoir revisiter le passé à travers des documents et à partir d'une approche multidimensionnelle : le réseau, l'iconographie, le temps et l'espace sont les points d'entrées. Bien que l’échelle spatiale de l’étude soit limitée à un quartier parisien, l'approche se veut ainsi exhaustive que possible. Le projet présente une certaine innovation en rassemblant ce vaste corpus de sources iconographiques et cartographiques numérisées, principalement constitué de photographies, d’estampes, de cartes et de plans issus d’institutions parisiennes, datant majoritairement du milieu du XIX\ieme  siècle. 

L'agrégation et la transformation de ce corpus en données, une fois nettoyées et formatées, ont généré un volume d'informations considérable à l’échelle du quartier, plus de 100 000 lignes pour un quartier de moins de $1~\text{km}^2$. Stockées dans une base de données relationnelle, présentée dans le deuxième chapitre, les données sont structurées par rapport à la notion de lieu qui occupe une place cardinale dans le projet Richelieu. Les étapes successives du traitement des données ont mis en lumière le besoin de leur structuration pour garantir la pérennité et l’accessibilité des sources historiques via l'application Web. Bien que son architecture client-serveur sur 3 niveaux est complexe au premier abord, elle est robuste et respecte les principes du Web. Elle permet notamment de développer au mieux la cartographie Web. 

Ainsi, la deuxième partie du mémoire a souligné l'importance de définir des méthodes de travail claires, des objectifs précis et des fonctionnalités adaptées pour la carte. Bien que les outils actuels de visualisation spatiale des données historiques offrent de nombreuses possibilités, une méthodologie rigoureuse s'avère indispensable pour orienter le développement. Ce travail demande la maîtrise de connaissances issues d'un large spectre de compétences techniques et théoriques : en géographie, en histoire de l'art, et surtout en informatique appliquée à ces domaines. Le principal défi reste la représentation de  la multidimensionnalité des données sur une carte sans la surcharger d'informations, afin de la rendre lisible et accessible. Le mémoire a présenté quelques outils utilisés par le projet Richelieu pour favoriser cette mise en place interdisciplinaire, même si les représentations de la mise en réseau et du temps sont mises de côté. 

La librairie Leaflet, présentée dans le quatrième chapitre, a joué un rôle central dans le développement du projet, grâce à des fonctionnalités telles que les filtres de données, la légende interactive et l'affichage d'images associées aux lieux. L'accent a été mis sur la connexion entre les données stockées et leur représentation cartographique dynamique, facilitée par l'API REST. Ainsi, les informations relatives au quartier sont accessibles à différentes échelles grâce aux niveaux de granularité, permettant à l'utilisateur de les consulter à l'échelle d'un ensemble architectural, d'une aile de bâtiment ou d’une parcelle. Le développement le plus avancé concerne la représentation de la densité d'information iconographique par lieu. Bien que certaines fonctionnalités soient encore au stade de prototype, elles sont des initiatives qui offrent des perspectives prometteuses pour l'avenir de la cartographie Web en histoire de l'art.

Celles-ci sont en effet abordées dans la troisième et dernière partie du mémoire. Le projet présente des défis liés à la pérennisation des données et le mémoire souligne l’importance de suivre le cycle de vie des données, de leur acquisition à leur archivage, pour garantir leur durabilité à long terme. Cette partie a surtout eu pour objectif de démontrer l'importance d'archiver pour assurer un retour vers le futur lointain afin que le projet participe activement au processus de patrimonialisation des plateformes numériques de recherche. 

Lorsque le projet sera terminé, sans doute en novembre prochain pour la phase actuelle, l'INHA offrira  une infrastructure technique capable de gérer efficacement les bases de données tout en offrant un accès continu aux chercheurs et au grand public. L'adoption des principes FAIR (Facile à trouver, Accessible, Interopérable, Réutilisable) par le projet,  garantit une première accessibilité et réutilisation des données dans le futur. Sur ce point, et selon le système de notation de l'inventeur du Web, \footcite{BERNERS-LEELinked2009} nous pouvons attribuer 3 étoiles au projet Richelieu : les données sont structurées (lisibles par machine), disponibles et ouvertes (avec une licence ouverte) sur le Web, et dans un format non propriétaire. 

L'approche interdisciplinaire et l'introduction de nouveaux outils de recherche dans le cadre du projet Richelieu confirment que la cartographie Web représente un \textit{nouveau} retour vers le futur. Ce projet contribue à redéfinir les frontières de l'histoire de l'art et à ouvrir la voie à une géohistoire de l'art numérique. À travers le prisme des \textit{Spatial Humanities} et des \textit{Digital Humanities}, il s'agit d'un champ de recherche transdisciplinaire qui transforme profondément les méthodes d'analyse des œuvres dans un contexte géospatial. Cette approche permet de revisiter et de représenter le passé de manière innovante, grâce à des corpus visuels et spatiaux désormais accessibles et compréhensibles. Le projet gagnerait en impact en affirmant sa présence tant auprès du grand public que de la communauté scientifique. Ces acteurs sont maintenant invités à explorer le site Web du projet Richelieu afin de déterminer si, d'une part, le quartier Richelieu constitue l'archétype de la capitale moderne en Europe au XIX\ieme siècle, et, d'autre part, si la plateforme cartographique Web incarne une méthodologie de recherche novatrice au XXI\ieme  siècle.

